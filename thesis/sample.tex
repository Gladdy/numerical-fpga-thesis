%*******************************************************************************
%*********************************** First Chapter *****************************
%*******************************************************************************

\chapter{SAMPLE CODE - Getting started}  %Title of the First Chapter

\ifpdf
\graphicspath{{P2-Results/Figs/Raster/}{P2-Results/Figs/PDF/}{P2-Results/Figs/}}
\else
\graphicspath{{P2-Results/Figs/Vector/}{P2-Results/Figs/}}
\fi

\section{What is loren ipsum? Title with math \texorpdfstring{$\sigma$}{[sigma]}} %Section - 1.1 

Lorem Ipsum is simply dummy text of the printing and typesetting industry (see 
Section~\ref{section1.3}). Lorem Ipsum~\citep{Aup91} has been the industry's 
standard dummy text ever since the 1500s, when an unknown printer took a galley 
of type and scrambled it to make a type specimen book. It has survived not only 
five centuries, but also the leap into electronic typesetting, remaining 
essentially unchanged. It was popularised in the 1960s with the release of 
Letraset sheets containing Lorem Ipsum passages, and more recently with desktop 
publishing software like Aldus PageMaker including versions of Lorem 
Ipsum~\citep{AAB95,Con90,LM65}.

The most famous equation in the world: $E^2 = (m_0c^2)^2 + (pc)^2$, which is 
known as the \textbf{energy-mass-momentum} relation as an in-line equation.

A {\em \LaTeX{} class file}\index{\LaTeX{} class file@LaTeX class file} is a file, which holds style information for a particular \LaTeX{}.


\begin{align}
CIF: \hspace*{5mm}F_0^j(a) = \frac{1}{2\pi \iota} \oint_{\gamma} \frac{F_0^j(z)}{z - a} dz
\end{align}

\nomenclature[z-cif]{$CIF$}{Cauchy's Integral Formula}                                % first letter Z is for Acronyms 
\nomenclature[a-F]{$F$}{complex function}                                                   % first letter A is for Roman symbols
\nomenclature[g-p]{$\pi$}{ $\simeq 3.14\ldots$}                                             % first letter G is for Greek Symbols
\nomenclature[g-i]{$\iota$}{unit imaginary number $\sqrt{-1}$}                      % first letter G is for Greek Symbols
\nomenclature[g-g]{$\gamma$}{a simply closed curve on a complex plane}  % first letter G is for Greek Symbols
\nomenclature[x-i]{$\oint_\gamma$}{integration around a curve $\gamma$} % first letter X is for Other Symbols
\nomenclature[r-j]{$j$}{superscript index}                                                       % first letter R is for superscripts
\nomenclature[s-0]{$0$}{subscript index}                                                        % first letter S is for subscripts


%********************************** %Second Section  *************************************
\section{Why do we use loren ipsum?} %Section - 1.2


It is a long established fact that a reader will be distracted by the readable content of a page when looking at its layout. The point of using Lorem Ipsum is that it has a more-or-less normal distribution of letters, as opposed to using `Content here, content here', making it look like readable English. Many desktop publishing packages and web page editors now use Lorem Ipsum as their default model text, and a search for `lorem ipsum' will uncover many web sites still in their infancy. Various versions have evolved over the years, sometimes by accident, sometimes on purpose (injected humour and the like).

%********************************** % Third Section  *************************************
\section{Where does it come from?}  %Section - 1.3 
\label{section1.3}

Contrary to popular belief, Lorem Ipsum is not simply random text. It has roots in a piece of classical Latin literature from 45 BC, making it over 2000 years old. Richard McClintock, a Latin professor at Hampden-Sydney College in Virginia, looked up one of the more obscure Latin words, consectetur, from a Lorem Ipsum passage, and going through the cites of the word in classical literature, discovered the undoubtable source. Lorem Ipsum comes from sections 1.10.32 and 1.10.33 of "de Finibus Bonorum et Malorum" (The Extremes of Good and Evil) by Cicero, written in 45 BC. This book is a treatise on the theory of ethics, very popular during the Renaissance. The first line of Lorem Ipsum, "Lorem ipsum dolor sit amet..", comes from a line in section 1.10.32.

The standard chunk of Lorem Ipsum used since the 1500s is reproduced below for those interested. Sections 1.10.32 and 1.10.33 from ``de Finibus Bonorum et Malorum" by Cicero are also reproduced in their exact original form, accompanied by English versions from the 1914 translation by H. Rackham

``Lorem ipsum dolor sit amet, consectetur adipisicing elit, sed do eiusmod tempor incididunt ut labore et dolore magna aliqua. Ut enim ad minim veniam, quis nostrud exercitation ullamco laboris nisi ut aliquip ex ea commodo consequat. Duis aute irure dolor in reprehenderit in voluptate velit esse cillum dolore eu fugiat nulla pariatur. Excepteur sint occaecat cupidatat non proident, sunt in culpa qui officia deserunt mollit anim id est laborum."

Section 1.10.32 of ``de Finibus Bonorum et Malorum", written by Cicero in 45 BC: ``Sed ut perspiciatis unde omnis iste natus error sit voluptatem accusantium doloremque laudantium, totam rem aperiam, eaque ipsa quae ab illo inventore veritatis et quasi architecto beatae vitae dicta sunt explicabo. Nemo enim ipsam voluptatem quia voluptas sit aspernatur aut odit aut fugit, sed quia consequuntur magni dolores eos qui ratione voluptatem sequi nesciunt. Neque porro quisquam est, qui dolorem ipsum quia dolor sit amet, consectetur, adipisci velit, sed quia non numquam eius modi tempora incidunt ut labore et dolore magnam aliquam quaerat voluptatem. Ut enim ad minima veniam, quis nostrum exercitationem ullam corporis suscipit laboriosam, nisi ut aliquid ex ea commodi consequatur? Quis autem vel eum iure reprehenderit qui in ea voluptate velit esse quam nihil molestiae consequatur, vel illum qui dolorem eum fugiat quo voluptas nulla pariatur?"

1914 translation by H. Rackham: ``But I must explain to you how all this mistaken idea of denouncing pleasure and praising pain was born and I will give you a complete account of the system, and expound the actual teachings of the great explorer of the truth, the master-builder of human happiness. No one rejects, dislikes, or avoids pleasure itself, because it is pleasure, but because those who do not know how to pursue pleasure rationally encounter consequences that are extremely painful. Nor again is there anyone who loves or pursues or desires to obtain pain of itself, because it is pain, but because occasionally circumstances occur in which toil and pain can procure him some great pleasure. To take a trivial example, which of us ever undertakes laborious physical exercise, except to obtain some advantage from it? But who has any right to find fault with a man who chooses to enjoy a pleasure that has no annoying consequences, or one who avoids a pain that produces no resultant pleasure?"

Section 1.10.33 of ``de Finibus Bonorum et Malorum", written by Cicero in 45 BC: ``At vero eos et accusamus et iusto odio dignissimos ducimus qui blanditiis praesentium voluptatum deleniti atque corrupti quos dolores et quas molestias excepturi sint occaecati cupiditate non provident, similique sunt in culpa qui officia deserunt mollitia animi, id est laborum et dolorum fuga. Et harum quidem rerum facilis est et expedita distinctio. Nam libero tempore, cum soluta nobis est eligendi optio cumque nihil impedit quo minus id quod maxime placeat facere possimus, omnis voluptas assumenda est, omnis dolor repellendus. Temporibus autem quibusdam et aut officiis debitis aut rerum necessitatibus saepe eveniet ut et voluptates repudiandae sint et molestiae non recusandae. Itaque earum rerum hic tenetur a sapiente delectus, ut aut reiciendis voluptatibus maiores alias consequatur aut perferendis doloribus asperiores repellat."

1914 translation by H. Rackham: ``On the other hand, we denounce with righteous indignation and dislike men who are so beguiled and demoralized by the charms of pleasure of the moment, so blinded by desire, that they cannot foresee the pain and trouble that are bound to ensue; and equal blame belongs to those who fail in their duty through weakness of will, which is the same as saying through shrinking from toil and pain. These cases are perfectly simple and easy to distinguish. In a free hour, when our power of choice is untrammelled and when nothing prevents our being able to do what we like best, every pleasure is to be welcomed and every pain avoided. But in certain circumstances and owing to the claims of duty or the obligations of business it will frequently occur that pleasures have to be repudiated and annoyances accepted. The wise man therefore always holds in these matters to this principle of selection: he rejects pleasures to secure other greater pleasures, or else he endures pains to avoid worse pains."

\section[Short title]{Reasonably long section title}

% Uncomment this line, when you have siunitx package loaded.
%The SI Units for dynamic viscosity is \si{\newton\second\per\metre\squared}.
I'm going to randomly include a picture Figure~\ref{fig:minion}.


If you have trouble viewing this document contact Krishna at: \href{mailto:kks32@cam.ac.uk}{kks32@cam.ac.uk} or raise an issue at \url{https://github.com/kks32/phd-thesis-template/}


\begin{figure}[htbp!] 
	\centering    
	\includegraphics[width=1.0\textwidth]{minion}
	\caption[Minion]{This is just a long figure caption for the minion in Despicable Me from Pixar}
	\label{fig:minion}
\end{figure}


\section*{Enumeration}
\begin{enumerate}
	\item The first topic is dull
	\item The second topic is duller
	\begin{enumerate}
		\item The first subtopic is silly
		\item The second subtopic is stupid
	\end{enumerate}
	\item The third topic is the dullest
\end{enumerate}

\section*{itemize}
\begin{itemize}
	\item The first topic is dull
	\item The second topic is duller
	\begin{itemize}
		\item The first subtopic is silly
		\item The second subtopic is stupid
	\end{itemize}
	\item The third topic is the dullest
\end{itemize}

\section*{description}
\begin{description}
	\item[The first topic] is dull
	\item[The second topic] is duller
	\begin{description}
		\item[The first subtopic] is silly
		\item[The second subtopic] is stupid
	\end{description}
	\item[The third topic] is the dullest
\end{description}


\clearpage

\tochide\section{Hidden section}
\textbf{Lorem ipsum dolor sit amet}, \textit{consectetur adipiscing elit}. In magna nisi, aliquam id blandit id, congue ac est. Fusce porta consequat leo. Proin feugiat at felis vel consectetur. Ut tempus ipsum sit amet congue posuere. Nulla varius rutrum quam. Donec sed purus luctus, faucibus velit id, ultrices sapien. Cras diam purus, tincidunt eget tristique ut, egestas quis nulla. Curabitur vel iaculis lectus. Nunc nulla urna, ultrices et eleifend in, accumsan ut erat. In ut ante leo. Aenean a lacinia nisl, sit amet ullamcorper dolor. Maecenas blandit, tortor ut scelerisque congue, velit diam volutpat metus, sed vestibulum eros justo ut nulla. Etiam nec ipsum non enim luctus porta in in massa. Cras arcu urna, malesuada ut tellus ut, pellentesque mollis risus.Morbi vel tortor imperdiet arcu auctor mattis sit amet eu nisi. Nulla gravida urna vel nisl egestas varius. Aliquam posuere ante quis malesuada dignissim. Mauris ultrices tristique eros, a dignissim nisl iaculis nec. Praesent dapibus tincidunt mauris nec tempor. Curabitur et consequat nisi. Quisque viverra egestas risus, ut sodales enim blandit at. Mauris quis odio nulla. Cras euismod turpis magna, in facilisis diam congue non. Mauris faucibus nisl a orci dictum, et tempus mi cursus.

Etiam elementum tristique lacus, sit amet eleifend nibh eleifend sed \footnote{My footnote goes blah blah blah! \dots}. Maecenas dapibu augue ut urna malesuada, non tempor nibh mollis. Donec sed sem sollicitudin, convallis velit aliquam, tincidunt diam. In eu venenatis lorem. Aliquam non augue porttitor tellus faucibus porta et nec ante. Proin sodales, libero vitae commodo sodales, dolor nisi cursus magna, non tincidunt ipsum nibh eget purus. Nam rutrum tincidunt arcu, tincidunt vulputate mi sagittis id. Proin et nisi nec orci tincidunt auctor et porta elit. Praesent eu dolor ac magna cursus euismod. Integer non dictum nunc.


\begin{landscape}
	
	\section*{Subplots}
	I can cite Wall-E (see Fig.~\ref{fig:WallE}) and Minions in despicable me (Fig.~\ref{fig:Minnion}) or I can cite the whole figure as Fig.~\ref{fig:animations}
	
	\begin{figure}
		\centering
		\begin{subfigure}[b]{0.3\textwidth}
			\includegraphics[width=\textwidth]{TomandJerry}
			\caption{Tom and Jerry}
			\label{fig:TomJerry}   
		\end{subfigure}             
		\begin{subfigure}[b]{0.3\textwidth}
			\includegraphics[width=\textwidth]{WallE}
			\caption{Wall-E}
			\label{fig:WallE}
		\end{subfigure}             
		\begin{subfigure}[b]{0.3\textwidth}
			\includegraphics[width=\textwidth]{minion}
			\caption{Minions}
			\label{fig:Minnion}
		\end{subfigure}
		\caption{Best Animations}
		\label{fig:animations}
	\end{figure}
	

\end{landscape}

\section{First section of the third chapter}
And now I begin my third chapter here \dots

And now to cite some more people~\citet{Rea85,Ancey1996}

\subsection{First subsection in the first section}
\dots and some more 

\subsection{Second subsection in the first section}
\dots and some more \dots

\subsubsection{First subsub section in the second subsection}
\dots and some more in the first subsub section otherwise it all looks the same
doesn't it? well we can add some text to it \dots

\subsection{Third subsection in the first section}
\dots and some more \dots

\subsubsection{First subsub section in the third subsection}
\dots and some more in the first subsub section otherwise it all looks the same
doesn't it? well we can add some text to it and some more and some more and
some more and some more and some more and some more and some more \dots

\subsubsection{Second subsub section in the third subsection}
\dots and some more in the first subsub section otherwise it all looks the same
doesn't it? well we can add some text to it \dots

\section{Second section of the third chapter}
and here I write more \dots

\section{The layout of formal tables}
This section has been modified from ``Publication quality tables in \LaTeX*''
by Simon Fear.

The layout of a table has been established over centuries of experience and 
should only be altered in extraordinary circumstances. 

When formatting a table, remember two simple guidelines at all times:

\begin{enumerate}
	\item Never, ever use vertical rules (lines).
	\item Never use double rules.
\end{enumerate}

These guidelines may seem extreme but I have
never found a good argument in favour of breaking them. For
example, if you feel that the information in the left half of
a table is so different from that on the right that it needs
to be separated by a vertical line, then you should use two
tables instead. Not everyone follows the second guideline:

There are three further guidelines worth mentioning here as they
are generally not known outside the circle of professional
typesetters and subeditors:

\begin{enumerate}\setcounter{enumi}{2}
	\item Put the units in the column heading (not in the body of
	the table).
	\item Always precede a decimal point by a digit; thus 0.1
	{\em not} just .1.
	\item Do not use `ditto' signs or any other such convention to
	repeat a previous value. In many circumstances a blank
	will serve just as well. If it won't, then repeat the value.
\end{enumerate}

A frequently seen mistake is to use `\textbackslash begin\{center\}' \dots `\textbackslash end\{center\}' inside a figure or table environment. This center environment can cause additional vertical space. If you want to avoid that just use `\textbackslash centering'


\begin{table}
	\caption{A badly formatted table}
	\centering
	\label{table:bad_table}
	\begin{tabular}{|l|c|c|c|c|}
		\hline 
		& \multicolumn{2}{c}{Species I} & \multicolumn{2}{c|}{Species II} \\ 
		\hline
		Dental measurement  & mean & SD  & mean & SD  \\ \hline 
		\hline
		I1MD & 6.23 & 0.91 & 5.2  & 0.7  \\
		\hline 
		I1LL & 7.48 & 0.56 & 8.7  & 0.71 \\
		\hline 
		I2MD & 3.99 & 0.63 & 4.22 & 0.54 \\
		\hline 
		I2LL & 6.81 & 0.02 & 6.66 & 0.01 \\
		\hline 
		CMD & 13.47 & 0.09 & 10.55 & 0.05 \\
		\hline 
		CBL & 11.88 & 0.05 & 13.11 & 0.04\\ 
		\hline 
	\end{tabular}
\end{table}

\begin{table}
	\caption{A nice looking table}
	\centering
	\label{table:nice_table}
	\begin{tabular}{l c c c c}
		\hline 
		\multirow{2}{*}{Dental measurement} & \multicolumn{2}{c}{Species I} & \multicolumn{2}{c}{Species II} \\ 
		\cline{2-5}
		& mean & SD  & mean & SD  \\ 
		\hline
		I1MD & 6.23 & 0.91 & 5.2  & 0.7  \\
		
		I1LL & 7.48 & 0.56 & 8.7  & 0.71 \\
		
		I2MD & 3.99 & 0.63 & 4.22 & 0.54 \\
		
		I2LL & 6.81 & 0.02 & 6.66 & 0.01 \\
		
		CMD & 13.47 & 0.09 & 10.55 & 0.05 \\
		
		CBL & 11.88 & 0.05 & 13.11 & 0.04\\ 
		\hline 
	\end{tabular}
\end{table}


\begin{table}
	\caption{Even better looking table using booktabs}
	\centering
	\label{table:good_table}
	\begin{tabular}{l c c c c}
		\toprule
		\multirow{2}{*}{Dental measurement} & \multicolumn{2}{c}{Species I} & \multicolumn{2}{c}{Species II} \\ 
		\cmidrule{2-5}
		& mean & SD  & mean & SD  \\ 
		\midrule
		I1MD & 6.23 & 0.91 & 5.2  & 0.7  \\
		
		I1LL & 7.48 & 0.56 & 8.7  & 0.71 \\
		
		I2MD & 3.99 & 0.63 & 4.22 & 0.54 \\
		
		I2LL & 6.81 & 0.02 & 6.66 & 0.01 \\
		
		CMD & 13.47 & 0.09 & 10.55 & 0.05 \\
		
		CBL & 11.88 & 0.05 & 13.11 & 0.04\\ 
		\bottomrule
	\end{tabular}
\end{table}
